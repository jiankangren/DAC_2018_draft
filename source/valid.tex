\section{Simulator Validation}	\label{sec:valid}
%
To validate the correctness and accuracy of EBeSS, we compare the execution parameters of EBeSS with a real NVP-based system prototype~\cite{}. 

% hardware parameters
\subsection{System configurations}
The configuration of the system prototype is listed in Tab.~\ref{}. 
This prototype is an solar-powered system, which contains a NVP and NVRF based radio frequency module.








To validate the reliability of EBeSS, we compare the experiment results of EBeSS with an existing NVP system~\cite{Liu2015Ambient} which adopts the 2-thres scheme. 
The system parameters are listed in Tab.~\ref{tab:valid-param}.
The validation compares the execution time and energy consumption with different power profiles and benchmarks.
The power profiles are real solar power traces in MIDC database from NREL Solar Radiation Research Laboratory~\cite{midc2015solar}.
By executing three benchmarks (\emph{fft}, \emph{queens} and \emph{sort}), Tab.~\ref{tab:valid-result} shows that, the difference between the simulation result (time/energy) and the measured result are is within 7.88\%/6.38\% and the average difference is 3.45\%/2.91\%.
Therefore, EBeSS is proved to have the ability of imitating the behavior of a real NVP based self-powered system prototype.

\begin{table}[t]
	\begin{center}
		%\vspace{-0pt}
		\caption{Parameter Settings of NVP System Prototype~\cite{Liu2015Ambient}.} \label{tab:valid-param}
		%\vspace{-5pt}
		\Fsize{8}
		\renewcommand{\arraystretch}{1.5}
		%\setlength{\tabcolsep}{1pt}
		\begin{tabular}{Ic|c|c|c|c|cI}
			\Xhline{1.2pt}
			Param.	& Oper. Freq.	& power	& Mem	& RegFile	& Cap.\\
			\Xhline{1pt}
			Value	& $1$MHz	& $0.16$mW	& $512$KB	& $128$B	& $10\mu F$\\
			\Xhline{1.2pt}
		\end{tabular}
		\vspace{-15pt}
	\end{center}
\end{table} 
\begin{table}[t]
	\begin{center}
	\caption{Comparison Between EBeSS and NVP Prototype.} \label{tab:valid-result}
	\vspace{-10pt}
	\Fsize{8}
	\renewcommand{\arraystretch}{1.5}
	\setlength{\tabcolsep}{1.8pt}
	\begin{tabular}{Ic|cIc|c|cIc|c|cIc|c|cI}
		\Xhline{1.2pt}
		\multicolumn{2}{IcI}{}	& \multicolumn{3}{cI}{FFT}		& \multicolumn{3}{cI}{QUEENS}	& \multicolumn{3}{cI}{SORT} \\
		\Xhline{1.2pt}

		Power	& Metrics		& Sim.	& Mea.	& Err.			& Sim.	& Mea.	& Err.			& Sim.	& Mea.	& Err. \\

		\Xhline{1pt}
		\multirow{2}[1.2]{*}{Trace 1}
			& engy./uJ		& 4.21	& 4.02	& 4.51\%		& 7.61	& 7.45	& 2.10\%		& 29.2	& 26.9	& 7.88\% \\
		\cline{2-11}
			& time/ms		& 39.2	& 36.7	& 6.38\%		& 77.4	& 74.3	& 4.01\%		& 266	& 253	& 4.89\% \\

		\Xhline{1pt}
		\multirow{2}[1.2]{*}{Trace 2}
			& engy./uJ		& 2.98	& 2.93	& 1.68\%		& 5.14	& 4.97	& 3.31\%		& 16.2	& 15.4	& 4.94\% \\
		\cline{2-11}
			& time/ms		& 21.2	& 20.6	& 2.83\%		& 51.1	& 49.6	& 2.94\%		& 172	& 166	& 3.49\% \\

		\Xhline{1pt}
		\multirow{2}[1.2]{*}{Trace 3}
			& engy./uJ		& 2.19	& 2.15	& 1.83\%		& 2.97	& 2.91	& 2.02\%		& 10.9	& 10.6	& 2.75\% \\
		\cline{2-11}
			& time/ms		& 14.5	& 14.4	& 0.69\%		& 24.5	& 24.5	& 0.00\%		& 91.1	& 90.2	& 0.99\% \\

		\Xhline{1.2pt}
	\end{tabular}
	\end{center}
	\vspace{-20pt}
\end{table} 