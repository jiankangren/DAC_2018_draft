% packages & styles
\usepackage{comment}
\usepackage{graphicx}     % Insert figures
\usepackage{setspace}      % line distance
\usepackage{bbding}        % yes and no
\usepackage{amssymb, amsmath}    % American Math Society, Symbols, Math equations
\usepackage{pifont}         % Add a circle around 'x' (\textcircled{x})
\usepackage{times}          % Times New Roman
\usepackage{booktabs}    % Tables, http://ddswhu.com/2014/08/24/9-essential-latex-packages/
\usepackage{multicol, multirow}     % Double collumns, rows
\usepackage{float}           % Figure Positions
\usepackage{makecell}    % Table Cell
%\usepackage{url}              % Add url in the references, http://blog.csdn.net/perfumekristy/article/details/8680045
\usepackage{threeparttable} % Three part table
%\usepackage{subfigure}   % Insert 2 figures side by side
\usepackage{algorithm,algpseudocode}    % Pseudo code
\usepackage{lettrine}       % A large letter at the fisrt letter of the article.
\usepackage{mathrsfs}    % Bold curlycue
%\usepackage{vector}       % Arrow on the letters: vector
%\usepackage[nocompress]{cite}   % Citation: [1,2,3,4]
%\usepackage{cite}           % Citation: [1-4]
%\usepackage[numbers,sort&compress]{natbib}
\usepackage[colorlinks, linkcolor=default, anchorcolor=default, citecolor=default]{hyperref} % generate a content index in PDF

\graphicspath{{images/}}


\begin{comment}
\DeclareMathOperator*{\argmin}{argmin}
\newtheorem{mydef}{Defination}
\newtheorem{lemma}{Lemma}
\newtheorem{assumption}{Assumption}
\newcommand{\block}[1]{
  \underbrace{\begin{matrix}1 & \cdots & 1\end{matrix}}_{#1}
}
\end{comment}

%\newtheorem{lemma}[theorem]{Lemma}

\begin{comment}
\setlength\abovedisplayskip{0pt}
\setlength\belowdisplayskip{0pt}
\setlength\footnotesep{0pt}
\setlength\floatsep{0pt}
\setlength\textfloatsep{0pt}
\end{comment}

%\newenvironment{pproof}[1][Proof]{\begin{trivlist}
%\item[\hskip \labelsep {\bfseries #1}]}{\end{trivlist}}
%\newenvironment{definition}[1][Definition]{\begin{trivlist}
%\item[\hskip \labelsep {\bfseries #1}]}{\end{trivlist}}
%\newenvironment{example}[1][Example]{\begin{trivlist}
%\item[\hskip \labelsep {\bfseries #1}]}{\end{trivlist}}
%\newenvironment{remark}[1][Remark]{\begin{trivlist}
%\item[\hskip \labelsep {\bfseries #1}]}{\end{trivlist}}

\begin{comment}
\newcommand{\vect}[1]{\boldsymbol{#1}}
\renewcommand{\arraystretch}{1.1}
%\renewcommand{\baselinestretch}{0.9} \normalsize
\newcommand{\tabincell}[2]{\begin{tabular}{@{}#1@{}}#2\end{tabular}}
\end{comment}

%##############################################################################
% Spacing
%##############################################################################
\begin{comment}
\newcommand{\setblstr}[2][.]{%
    \renewcommand{\baselinestretch}{#2}%
    \ifx#1.\else\setarrstr  {#1}\fi%
}

\newcommand{\setarrstr}[1]{%
    \renewcommand{\arraystretch}{#1}%
}

\newenvironment{blstr}[2][.]{%
    \setblstr[#1]{#2}%
}{%
}
\setlength{\textfloatsep} {12pt plus 2pt minus 2pt}
\end{comment}
%##############################################################################
% Comments
%##############################################################################
\begin{comment}
% Hiding
\excludecomment{Hidden}
\newcommand{\Hide}[1]{}
\newcommand{\ShowHidden}{%
    \includecomment{Hidden}%
    \renewcommand{\Hide}[1]{%
        ##1%
    }%
}

\specialcomment{Notes}{%
    \noindent$\Rightarrow\Rightarrow$%
}{%
    $\Leftarrow\Leftarrow$%
}

\providecommand{\Note}[1]{%
    \noindent{\bf$\Rightarrow\Rightarrow$#1$\Leftarrow\Leftarrow$}%
}

\newcommand{\HideNotes}{%
    \excludecomment{Notes}%
    \renewcommand{\Note}[1]{}%
}
\end{comment}
%##############################################################################
% Fonts
%##############################################################################
%\begin{comment}
\newcommand{\Fsize}[2][.]{%
    \ifthenelse{\equal{#1}{.}}{
        \fontsize{#2}{#2}%
    }{%
        \fontsize{#2}{#1}%
    }%
    \selectfont%
}
%\DeclareMathSizes{10}{18}{12}{8}
%\end{comment}
%##############################################################################
% Tables
%##############################################################################
%\begin{comment}
\newcommand{\Drop}[1]{%
    \multirow{2}{*}{#1}%
}
\newcolumntype{I}{!{\vrule width 0.8pt}}
\newenvironment{newitemize}{
    \begin{itemize}
      \setlength{\itemsep}{0pt}
      \setlength{\parskip}{0pt}
      \setlength{\parsep}{0pt}
}{\end{itemize}}
\setlength{\intextsep}{5pt plus 3pt minus 3pt}
%\setlength\textfloatsep{1.25\baselineskip plus 1pt minus 1pt}
%\end{comment}
%##############################################################################
% Algo
%##############################################################################
%\RestyleAlgo{ruled}

%\renewcommand{\algorithmicrequire}{\textbf{Input:}} % Use Input in the format of Algorithm
%\renewcommand{\algorithmicensure}{\textbf{Output:}} % Use Output in the format of Algorithm 
